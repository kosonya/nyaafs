%Данный текст распространяется на правах общественного достояния. Вы можете реализовать эту файловую систему за нас даже без нашего ведома.
% $Id: tz.tex 5 2009-03-23 09:53:25Z Maxim.Kovalev $
\documentclass[pdftex,a4paper,11pt]{article}
\usepackage{ucs}
\usepackage[utf8x]{inputenc}
\usepackage{fontenc}
\usepackage[pdftex,bookmarks,colorlinks,unicode=true]{hyperref}
\usepackage[english,russian]{babel}
\usepackage{graphicx}
\usepackage{wrapfig} 
\author{Ковалев~М.М., Бадаев В.А., группа C-44}
\title{Реляционная файловая система nyaafs, техническое задание}


\begin{document}
\maketitle

\hypersetup{linkcolor=black}
\tableofcontents
\hypersetup{linkcolor=red}

\section{Общее описание}
Реляционная файловая система nyaafs --- это виртуальная файловая система, реализуемая через модуль Fuse, работающая поверх любой другой POSIX-совместимой файловой системы, которая предоставляет интерфейс к содержимому диска как в виде обычного дерева директорий, так и в виде содержимого реляционной базы данных.

Реализация дерева директорий также осуществляется средствами базы данных. В базе данных хранится только структура файловой системы, сами файлы хранятся классическим образом.

Любое свойство файла, в том числе директория, в которой он лежит, представляется отношением в базе данных между идентификатором файла и указанным свойством.

\section{API файловой системы}

Данная файловая система не является модулем ядра и реализует API, требуемый модулем fuse. Представленные функции:

\begin{description}

\item[getattr(путь к файлу)] --- возвращает аргументы файла, описанные в стандарте POSIX.

\begin{description}

\item[st\_ino] --- inode файла.
\item[st\_dev, st\_blksize] --- игнорируются fuse, но должны быть определены.
\item[st\_mode] --- тип объекта и права доступа.
\item[st\_nlink] --- количество жёстких ссылок на файл.
\item[st\_uid] --- ID владельца файла.
\item[st\_gid] --- ID группы, владельца файла.
\item[st\_rdev] --- ID устройства (если доступно).
\item[st\_size] --- Размер файла.
\item[st\_blocks] --- количество блоков, выделенных под файл.
\item[st\_atime] --- время последнего доступа к файлу.
\item[st\_mtime] --- время последнего изменения файла.
\item[st\_ctime] --- время последнего изменения мета-данных файла.

\end{description}

Если файла по указанному пути не существует, функция возвращает ошибку.

\item[readlink(путь к файлу)] --- если файл является символьной ссылкой, возвращает цель, на которую эта ссылка указывает.

\item[mknod(путь, режим, ID устройства)] --- Создание объекта по указанному пути. Режим задаёт тип и права доступа, а ID устройства указывается только в том случае, если путь является устройством.

\item[mkdir(путь, режим)] --- Создание директории. Режим задаёт права доступа.

\item[readdir(путь, смещение)] --- прочитать содержимое директории по указанному пути с указанным смещением.

\item[rmdir(путь)] --- удалить директорию по указанному пути.

\item[chmod(путь, режим)] --- меняет права доступа к файлу с указанным путём на указанные режимом.

\item[chown(путь, владелец, группа)] -- меняет владельца и группу указанного файла на указанные.

\item[unlink(путь)] --- Удаляет любой объект, кроме директории.

\item[symlink(цель, имя)] --- содаёт символьную ссылку с указанным именем на указанную цель.

\item[rename(старое имя, новое имя)] --- переименовывает объект.

\item[link(цель, имя)] --- создаёт жёсткую ссылку с укзанным именем на указанную цель.

\item[open(путь, флаги)] --- открывает файл по указанному пути с указанными флагами.

\item[create(путь, флаги, режим)] --- создаёт файл по указанному пути с указанным режимом. Флаги управляют работой функции.

\item[read(дескриптор, длина, смещение)] --- чтение участка указанной длины, начиная с указанного смещения, из файла с указанным дескриптором. Возвращает прочитанный участок.

\item[write(дескриптор, участок, смещение)] --- пишет переданный участок в файл с указанным дескриптором, начиная с указанного смещения.

\item[release(дескриптор, флаги)] --- закрытие файла.

\item[flush(дексриптор)] --- запись изменений в файле с указанным дескриптором.
\end{description}

\subsection{Особенности реализации}

Файловая система реализуется на языке python с использованием SQLite в качестве СУБД. Файл реализуется как отельный класс с доступным интерфейсом.

\section{Реализация дерева директорий}

Каждая директория имеет ID, являющийся уникальным по всей файловой системе, и имя, которое может не быть уникальным.

В одной директории не могут находится файл и директория с совпадающими именами.

Для реализации дерева директорий существуют следующие таблицы:

\begin{description}

\item[Имена директорий] --- определяет для каждого ID директории иммя, которое ему соотвествует.
\item[Вложенность директорий] --- для каждого ID дректории определяет ID директории, в которую первая вложена.
\item[Директория файла] --- Для каждого ID файла определяет ID директории, в которой он находится.
\end{description}

Когда стоит задача найти файл или директорию по указанному пути, СУБД просматривает таблицы вложенности директорий и их имён для оппредения ID предпоследнего элемента пути, который гарантированно является директорией.

Далее система по таблицам имён файлов и директорий определяет, является ли последний элемент пути файлом или директорий, и возвращает его ID.

\subsection{Пользовательские деревья}

Файловая система также предоставляет пользователю возможность создавать свои деревья на аналогичном принципе.

\section{Реализация реляционной модели}

Файловая система хранит все атрибуты файлов и все деревья. К ним пользователь может получить доступ двумя путями: через фукнцию получения атрибутов и через запрос к базе данных.

Все запросы, путь в которых начинается с символов \frqq{}/.filesystem/bd/\flqq{}, считаются не запросами к дереву директорий, а запросами к базе данный. После этих символов следует обычный SQL запрос. Если идёт запрос на чтение, то файловая система возвращает список результатов в виде директории. Результаты других запросов пишутся в специальный log-файл.
\end{document}
